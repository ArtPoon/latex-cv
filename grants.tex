% !TEX root = artpoon-CV.tex

\section{Grants}

Only grants on which I am Principal Investigator/Applicant, Co-applicant or Co-investigator are listed.
These lists omit grants in which I was included as a Collaborator or a Consultant without direct research or salary support.


\subsection {Pending awarded}

\cventry{2024/6 -- 2029/5}
{$\bigstar$ Principal Applicant}
{NSERC Discovery Grant}
{Modeling the evolution of virus genomes}
{Total funding: \$280,000}
{
This is a successful renewal of my previous NSERC Discovery Grant. 
The objectives of this application are: 
(1) to develop a network-based model of virus genome evolution;
(2) to characterize patterns of selection among genes of RNA virus genomes, and;
(3) to model the persistence of virus genomes under rapid mutation rates.\\
}


\subsection {Active}


\cventry{2022/10 -- 2027/9}
{$\bigstar$ Principal Applicant}
{CIHR Project Grant}
{Adapting genetic clustering methods to SARS-CoV-2}
{Total funding: \$596,700} %\$779,105}
{
The objectives of this proposal are to: (1) adapt methods from network science to partition large databases of SARS-CoV-2 genomes into clusters that are calibrated to measure the impact of age, location and other risk factors on transmission rates; (2) develop fast, approximate methods to extract epidemiological information, such as the number of unsampled infections, from cluster-based trees updated in real time; and (3) adapt a method from dynamic social network analysis to reconstruct the role of recombination (the exchange of fragments between genomes) in the evolutionary history of coronaviruses.\\
}


\cventry{2021/8 -- 2026/4}
{Co-investigator}
{NIH ZAI1}  % 2R01AI049170-14A1
{REACH: Research Enterprise to Advance a Cure for HIV}
{Total funding: \$5,683,679 USD}
{
The objective of this multi-center study is to characterize the relationship between the persistent HIV reservoir, CD8+ T-cells, and rebound virus, to overcome barriers to eradication of HIV reservoirs by the immune response. 
My role as co-project lead for Research Focus (RF) 1, Aim 3 is to contribute phylogenetic expertise in reconstructing the integration dates of HIV provirus in the latent reservoir. 
As a co-investigator based at Western, I receive about \$50,000 CAD from the subgrant held by Dr.~Jessica Prodger.
}


\cventry{2020/5 -- 2025/4}
{Co-investigator}
{NIH R01}  % 2R01AI049170-14A1
{Determinants of HIV transmission fitness}
{Total funding: \$411,631 USD}
{
This grant is held by my colleague and frequent collaborator Dr.~Eric Arts (Western University) as principal investigator.
My role will be to provide bioinformatic support for the phylogenetic analysis of HIV-1 transmission variants sampled in the study.
As a co-investigator, I hold a subgrant from this grant for about \$45,000 per year.
}



\cventry{2018/6 -- 2024/5}
{$\bigstar$ Principal Applicant}
{NSERC Discovery Grant}
{Modeling the evolution of virus genomes}
{Total funding: \$215,040} %\$179,200} % RGPIN-2018-05516
{
The objectives of this application are (1) to develop empirical models of sequence insertions in virus genomes; (2) to develop simulation-based methods for measuring the impact of selection on rates of evolution within overlapping reading frames in viruses; and (3) to develop models of gene birth-death in virus genomes.\\
This grant was extended by NSERC by one additional year to lessen the impact of COVID-19 on research programs supported by the agency.
}





\vspace{1em}

\subsection {Completed}



\cventry{2018/4 -- 2023/3}
{$\bigstar$ Principal Applicant}
{CIHR Project Grant} % PJT-155990
{Phylodynamics of HIV within hosts}
{Total funding: \$451,350}
{
The objectives of this project are to develop and validate new simulation-based methods to fit dynamic models to within-host HIV-1 sequence variation, and to use these methods to elucidate the mechanisms that sustain the latent viral reservoir and to reconstruct the migration of the virus between anatomical and cellular compartments of the host.
This application is a resubmission of my Project Grant that received one year of bridge funding.\\ 
}


\cventry{2018/4 -- 2023/3}
{$\bigstar$ Principal Applicant}
{CIHR Project Grant} % PJT-156178
{Development, evaluation and implementation of genetic clustering methods for the real-time molecular surveillance of HIV outbreaks}
{Total funding: \$401,625}
{
This research proposal will support the development of a completely new method for genetic clustering in application to HIV molecular epidemiology; 
to evaluate clustering methods in the context of HIV prevention with computer simulations and the retrospective analysis of population databases; 
and to support the implementation and evaluation of real-time monitoring systems based on these methods in Uganda and Malaysia.\\
}



\cventry{2018/10 -- 2023/9}
{Co-investigator}
{CIHR Project Grant}  % PJT-159625
{Integrated phylogenetic, molecular and functional analyses of the within-host latent HIV reservoir}
{Total funding: \$1,141,767}
{
This grant is held by Dr.~Zabrina Brumme and Dr.~Jeffrey Joy of the BC Centre for Excellence in HIV/AIDS.
The objectives of the grant are to characterize the dynamics and adaptation of HIV within the latent reservoir, and to functionally characterize HIV reservoir sequences \textit{in vitro}.
My role is to contribute to the bioinformatic analyses of data derived from this study.
}

\cventry{2018/4 -- 2021/3}
{Lead investigator (1 of 6)}
{Canadian Statistical Sciences Institute (CANSSI) Collaborative Research Team Project}
{Statistical   methods   for   challenging   problems   in   public   health   microbiology}
{Total funding: \$180,000 (portion received: \$30,000)}
{Principal Applicants: Dr.~Leonid Chindelevich and Dr.~Alexandre Bouchard-C\^ot\'e\\
This team grant combines lead investigators from 5 different universities throughout Canada (Simon Fraser University,  Centre Hospitalier Universitaire Sainte-Justine, University of British Columbia, Western University, and Universit\'e de Montreal).
The aims of this grant are to develop computational methods to call genomic variants of pathogens by fitting evolutionary models; to develop machine learning classifiers to combine multiple inputs to predict drug resistance from whole genome sequencing data; and to detect genotype-phenotype associations in bacteria.\\
}



\cventry{2018/6 -- 2020/5}
{$\bigstar$ Co-Principal Investigator}
{National Institutes of Health (NIH) R21}  % 1R21AI127029-01A1
{Genetics, dynamics and fitness of the HIV-1 latent reservoir}
{Total funding: \$300,200 USD}
{
I am a co-PI on this grant along with Dr.~Zabrina Brumme and Dr.~Mark Brockman of Simon Fraser University (SFU).
I will receive about \$80,000 USD of the budget through a sub-contract between SFU and Western.
The objectives of this grant are to develop a new phylogenetic framework for dating HIV reservoir sequences within a host, and to use this framework to examine the dynamics of the HIV reservoir within individuals on long-term suppressive antiretroviral treatment.
}

\cventry{2017/9 -- 2019/3}
{Co-investigator}
{CIHR Operating Grant: Innovative Biomedical and Clinical HIV/AIDS Research}
{Developing the ACT-VEC as an HIV therapeutic vaccine and cure approach}
{Total funding: \$250,000 (portion received: \$86,445)}
{
Principal Applicant: Dr.~Eric Arts (and 3 others)\\
The objective of this application is to develop vectors containing near-complete HIV genomes as a heterologous subtype-specific vaccine to induce the reactivation of infected CD4+ T cells from the latent reservoir, as a primary component of an HIV cure strategy.
My role will be to design and implement a bioinformatic/phylogenetic algorithm for selecting patient-derived HIV genomes to incorporate into the vaccines.\\
}


\cventry{2016/10 -- 2019/3}
{$\bigstar$ Principal Investigator}  % OGI-131
{Genome Canada/CIHR: Bioinformatics and Computational Biology}
{Kamphir: a versatile framework to fit models to tree shapes}
{Total funding: \$205,365}
{
The objective of this grant is the development and validation of a software module in R for fitting epidemic and diversification models to the shapes of phylogenetic trees, as an innovative means of studying the spread and evolution of viruses.\\
}



\cventry{2017/10 -- 2018/9}
{$\bigstar$ Principal Investigator}
{CIHR Project Grant: Bridge Funding}  % PJT-153391
{Phylodynamics of HIV within hosts}
{Total funding: \$100,000}
{
The main objectives of this grant are to develop a new branch of phylodynamics for studying the evolution of HIV within hosts, and to use these methods to test hypotheses on the maintenance of the latent HIV reservoir, and to study the compartmentalization of HIV evolution between the blood and genital tract.\\
}



\cventry{2014/1 -- 2018/12}
{Co-investigator}
{CIHR Team Grant: HIV Cure Research}
{The Canadian HIV Cure Research Enterprise (CanCURE)}
{Total funding: \$8,760,000 (portion received \$45,391)}
{
Principal Applicant: Dr.~Eric Cohen (and 8 others)\\
The objective of this grant is to investigate the role of myeloid cells in maintaining the latent HIV reservoir.
My role is to provide bioinformatic support for the basic science component led by Drs.~Zabrina Brumme and Mark Brockman.\\
}

\cventry{2013/6 -- 2018/2}
{Co-investigator}
{National Institutes of Health (U.S.) R01}
{Seek and treat for the optimal prevention of HIV \& AIDS in BC}
{Total funding: \$530,331}
{
The objective of this grant is to support the implementation of strategic `test and treat' initiatives for HIV prevention in British Columbia.
My role was to provide bioinformatic support, including the phylogenetic and geographic analysis of HIV incidence trends based on the population HIV treatment database.\\
}



\cventry{2013/01 -- 2017/12}
{$\bigstar$ Co-applicant}
{Genome Canada: Genomics and Personalized Health}
{Viral and Human Genetic Predictors of Response to HIV Therapies}
{Total funding: \$4,100,000 (portion received: \$69,908)}
{
Project Leaders: Dr.~P.~Richard Harrigan, Dr.~Julio Montaner\\
The main objectives of this grant was to develop and validate a clinical next-generation sequencing (NGS) pipeline for diagnosing HIV drug resistance;
and to build a real-time monitoring system of transmitted HIV drug resistance.
My role was to design and implement both the clinical NGS pipeline and the real-time system, and eventually to supervise a team of developers to enhance and maintain these systems as lead developer.
Out of three Project Deliverables, I made major contributions to the first, and was entirely responsible for the second.\\
}


\cventry{2012/9 -- 2017/8}
{Co-investigator}
{National Institutes of Health (U.S.) R01}
{Impacts of universal access to HIV/AIDS care among HIV+ injection drug users}
{Total funding: \$2,424,355 (portion received: \$98,160)}
{
Principal Investigator: Dr.~Evan Wood\\
This application requested continued funding support for the Vancouver Injection Drug User Study (VIDUS) and ACCESS cohorts in British Columbia, for the further evaluation of increasing access to antiretroviral treatment and harm reduction resources in Vancouver.
My role was to provide phylogenetic support for characterizing the effects of these initiatives on transmission rates.\\
}


\cventry{2014/7 -- 2017/7}
{Co-investigator}
{Bill \& Melinda Gates Foundation Grant}
{Biostatistical, Computational Biology, and Mathematical Modeling for the Assessment of
Immune Correlates of Protection in the HVTN 701 and 702 Efficacy Trials of South Africa}
{Total funding: \$1,051,333 (portion received: \$8,000)}
{
Principal Investigator: Dr.~Peter B.~Gilbert
The purpose of this project was to develop, validate and apply phylogenetic methods to characterize the dynamics and adaptation of HIV in individuals who had become infected while participating in the vaccine efficacy trials HVTN 701 and 702.
My role was to implement some of the standard phylogenetic methods into a Python module to facilitate batch processing and testing.\\
}


\cventry{2012/10 -- 2015/9}
{Co-investigator}
{CIHR Operating Grant}
{Measuring mitochondrial aging, application to HIV infection and therapy}
{Total funding: \$330,150 (portion received: \$0)}
{
Principal Investigator: Dr.~H\'el\`ene C\^ot\'e\\
My role in this project was to assist in the analysis of next-generation sequence data that was collected to discriminate between the \textit{de novo} emergence and clonal expansion of mtDNA variants within patients undergoing antiretroviral treatment.\\
}

\cventry{2011/10 -- 2014/9}
{Co-investigator}
{CIHR Operating Grant}
{HIV adaptation to immune selection pressures: historic trends and future implications}
{Total funding: \$318,132 (portion received: \$5,000)}
{
Principal Investigator: Dr.~Zabrina Brumme\\
My role was to perform phylogenetic, molecular clock and ancestral reconstruction analyses on historic and modern HIV sequences.\\
}

\cventry{2011/10 -- 2014/9}
{Co-investigator}
{CIHR Operating Grant}
{Hepatitis C virus transmission dynamics among injection drug users}
{Total funding: \$336,569}
{
Principal Investigator: Jason Grebely\\
My role was to provide phylogenetic and genetic analysis software tools and expertise on their usage and interpretation of results in the context of HCV epidemics.\\
}

\cventry{2011/4 -- 2014/3}
{$\bigstar$ Principal Investigator}
{CIHR Operating Grant}
{Reconstructing within-host evolution of HIV-1 from next-generation sequencing data}
{Total funding: \$169,441}
{
The objective of this grant was to develop phylogenetic methods for the analysis of next-generation sequence data to reconstruct the evolution of HIV within hosts.
Specific objectives were to develop methods to estimate the date of HIV infection from sequence variation, and to map the emergence of specific variants associated with pathogenesis to the timeline of infection.\\
}


\cventry{2012/3 -- 2013/2}
{$\bigstar$ Principal Applicant}
{NIH R01 Administrative Supplement}
{Evaluating the natural history of injection drug use}
{Total funding: \$81,687}
{
Principal Investigator of parent grant: Dr.~Thomas Kerr\\
The purpose of this supplement was to develop methods to estimate the dates of HIV infection from bulk sequence chromatograms, to develop next-generation sequence (NGS) processing pipelines, and methods to estimate dates of HIV infection from NGS data sets.\\
}


