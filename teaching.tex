% !TEX root = artpoon-CV.tex


\section {Teaching}

Only teaching contributions at Western University are listed.\\

\cventry{2022/9 -- 2022/12}{Introduction to Medical Bioinformatics}{MBI3100A}{Department of Biology/Pathology \& Laboratory Medicine}{Western University}{
\begin{itemize}
  \item As one of four instructors, I am responsible for 6 weeks of lectures (2 lectures per week, 22 contact hours total).
  \item Enrolment: 23 students.
  \item This year we nearly doubled our enrolment over the previous year.  I adapted the curriculum in response to feedback from the previous year, such as placing greater emphasis on generic clustering analysis in bioinformatics instead of applications in epidemiology.
\end{itemize}
}


\cventry{2022/1 -- 2022/4}{Applied Bioinformatics}{BIOL9919B/PATH9577B}{Department of Biology/Pathology \& Laboratory Medicine}{Western University}{
\begin{itemize}
  \item As one of four instructors, I was responsible for 3 weeks of lectures (4 lectures).  I also attended and evaluated student presentations for 5 days.  In total, I had 18 contact hours.
  \item Enrolment: 14 students.
  \item This year, we decided to retain the online format developed in the previous year.  I shared responsibilities teaching the R programming language component of this course with the new instructor Dr.~Christina Castellani.
\end{itemize}
}

\cventry{2022/1 -- 2022/4}{Bioinformatics of Infectious Disease}{MICB4750G}{Department of Microbiology \& Immunology}{Western University}{
\begin{itemize}
\item Course coordinator and instructor (49 contact hours --- 2h lecture, 2h lab per week).
\item Enrolment: 15 students.
\item Because I had moved about one-third of the course materials to the new third-year course (Introduction to Medical Bioinformatics), I expanded and incorporated new material into this course's lectures, including a greatly expanded section on the ethics of bioinformatics.  In addition, I updated all lab assignments to focus even more closely on the analysis of SARS-CoV-2 genomes.
\end{itemize}
}


\cventry{2021/9 -- 2021/12}{Introduction to Medical Bioinformatics}{MBI3100A}{Department of Biology/Pathology \& Laboratory Medicine}{Western University}{
\begin{itemize}
  \item As one of four instructors, I was responsible for 6 weeks of instruction (one lecture and one lab session per week, 24 contact hours total).
  \item Enrolment: 12 students.
  \item In 2020-2021, I designed and implemented a new third-year undergraduate course, Introduction to Medical Bioinformatics (MBI 3100) and coordinated with three other faculty members to design the syllabus and methods of evaluation.
\end{itemize}
}


\cventry{2021/1 -- 2021/4}{Applied Bioinformatics}{BIOL9919B/PATH9577B}{Department of Biology/Pathology \& Laboratory Medicine}{Western University}{
\begin{itemize}
  \item As one of two instructors, I was responsible for 4 weeks of lectures (2 lectures per week, 18 contact hours total).
  \item Enrolment: 15 students.
  \item I redesigned my portion of the course to be taught entirely online.  Part of this involved redesigning the methods of evaluation, where we switched to oral examinations to assess how well students had achieved the learning objectives.
  %\item As course coordinator for Pathology, I led the reconfiguration of the course from a one lecture to two lecture per week format with an increased number of assignments, in response to feedback from students in the previous year.
  %\item I anticipated the `lockdown' of the university in response to the growing SARS-CoV-2 pandemic and co-ordinated the migration of the class into an online format on March 18, 2020.
\end{itemize}
}


\cventry{2021/1 -- 2021/4}{Bioinformatics of infectious disease}{MICB4750G}{Department of Microbiology \& Immunology}{Western University}{
\begin{itemize}
\item Course coordinator and instructor (49 contact hours --- 2h lecture, 2h lab per week).
\item Enrolment: 20 students.
\item I spent about two weeks (40 hours) creating a JavaScript web application to emulate the functionality of a Java program called BEAUti, which is used to prepare data for analysis with a popular phylogenetics software package called BEAST. 
This became necessary because Western Technology Services (WTS) refused to install Java on the GenLab computers that we used for the MIMM4750G lab. 
I wrote about 2,000 lines of code in order to create a replacement for the class, which I released as an open source project at https://github.com/ArtPoon/BelleJS.
%\item Revised curriculum:  I developed new lecture materials around the analysis and interpretation of SARS-CoV-2 genomes in response to the initial spread of the virus in China in January, 2020.  In addition, I modified lab 3 (January 21), to have students analyze the newly published genome sequence using BLAST.
\end{itemize}
}

\cventry{2020/7 -- 2021/4}{Seminar Research Project}{MHI4980E}{Department of Pathology}{Western University}{
\begin{itemize}
\item Course coordinator and instructor (26 contact hours).
\item Solicited research proposals from faculty across multiple Departments of Western University.
\item Matched students to proposals/labs following interviews and feedback from both potential supervisors and students.
\item Redeveloped course seminars to focus on practical research skills, \textit{e.g.}, scientific writing, data visualization, identifying journals of high and low quality.
\end{itemize}
}


\cventry{2020/1 -- 2020/4}{Bioinformatics of infectious disease}{MICB4750G}{Department of Microbiology \& Immunology}{Western University}{
\begin{itemize}
\item Course coordinator and instructor (52 contact hours --- 2h lecture, 2h lab per week).
\item Enrolment: 16 students.
\item Revised curriculum:  I developed new lecture materials around the analysis and interpretation of SARS-CoV-2 genomes in response to the initial spread of the virus in China in January, 2020.  In addition, I modified lab 3 (January 21), to have students analyze the newly published genome sequence using BLAST.
\end{itemize}
}


\cventry{2020/1 -- 2020/4}{Applied Bioinformatics}{BIOL9919B/PATH9577B}{Department of Biology/Pathology \& Laboratory Medicine}{Western University}{
\begin{itemize}
  \item As one of three instructors, I was responsible for 4 weeks of lectures (2 lectures per week, 16 contact hours total).
  \item Enrolment: 15 students.
  \item As course coordinator for Pathology, I led the reconfiguration of the course from a one lecture to two lecture per week format with an increased number of assignments, in response to feedback from students in the previous year.
  \item I anticipated the `lockdown' of the university in response to the growing SARS-CoV-2 pandemic and co-ordinated the migration of the class into an online format on March 18, 2020.
\end{itemize}
}

\cventry{2019/2 -- 2019/2}{Advanced Genetics}{BIOL3595B}{Department of Biology}{Western University}{
\begin{itemize}
  \item Guest lecture (1 contact hour).
  \item Fielded questions from students on concepts in bioinformatics and its application to the study of HIV-1.
\end{itemize}
}

\cventry{2019/1 -- 2019/4}{Applied Bioinformatics}{BIOL9919B/PATH9577B}{Department of Biology/Pathology \& Laboratory Medicine}{Western University}{
\begin{itemize}
  \item Graduate course that I developed as PATH9577Q has been adopted by the Department of Biology at Western for an expanded and cross-listed graduate course.
  \item I am one of three instructors.  I am responsible for 5 weeks of lectures (15 contact hours total).
  \item Enrolled 30 (max.~capacity) students in first year offered, with additional students waitlisted. 
  \item I gave the course materials that I had developed for my Python course to one of the co-instructors, and am developing new course materials on R.
\end{itemize}
}

\cventry{2019/1 -- 2019/4}{Bioinformatics of infectious disease}{MICB4750G}{Department of Microbiology \& Immunology}{Western University}{
\begin{itemize}
\item Development of materials for new course, 2017-12-09 to present (about 80 hours to date).
\item Teaching innovation: developing d3/JavaScript animations as learning tools.
\item Teaching innovation: development and release of online course materials using Jekyll/Markdown framework under a Creative Commons license.
\end{itemize}
}

\cventry{2019/1 -- 2019/1}{Biological and Social Determinants of Infectious Disease}{MICB3500B}{Department of Microbiology \& Immunology}{Western University}
{\begin{itemize}
\item Guest lecture (2 contact hours)
\end{itemize}}


\cventry{2018/5 -- 2018/7}{Bioinformatic data processing with Python}{PATH9577Q}{Department of Pathology \& Laboratory Medicine}{Western University}{
\begin{itemize}
\item Renewing curriculum based on student evaluations from previous year.
\end{itemize}}

\cventry{2018/1 -- 2018/1}{Biological and Social Determinants of Infectious Disease}{MICB3500B}{Department of Microbiology \& Immunology}{Western University}
{\begin{itemize}
\item Guest lecture (2 contact hours)
\end{itemize}}

\cventry{2017/5 -- 2017/6}{Bioinformatic data processing with Python}{PATH9577Q}{Department of Pathology \& Laboratory Medicine}{Western University}{
\begin{itemize}
\item Instructor and coordinator (18 contact hours).
\item Developed materials for new course as Markdown and JavaScript-based online content  2016-09-20 to 2017-06-26 (about 30 hours). 
\item Course materials released into public domain at \url{https://github.com/PoonLab/courses/tree/master/PATH9577Q}.
\end{itemize}}

\cventry{2017/1 -- 2017/1}{Biological and Social Determinants of Infectious Disease}{MICB3500B}{Department of Microbiology \& Immunology}{Western University}
{\begin{itemize}
\item Guest lecture (2 contact hours)
\end{itemize}}

\cventry{2017/3 -- 2017/4}{Biosystematics and Phylogenetics}{BIOL4289B}{Department of Biology}{Western University}
{\begin{itemize}
\item Guest lectures (2 contact hours)
\end{itemize}
}


