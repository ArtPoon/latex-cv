%% Copyright 2006-2015 Xavier Danaux (xdanaux@gmail.com).
%
% This work may be distributed and/or modified under the
% conditions of the LaTeX Project Public License version 1.3c,
% available at http://www.latex-project.org/lppl/.

\documentclass[11pt]{moderncv}

\usepackage[sfdefault]{cabin}

\moderncvstyle{casual}
\moderncvcolor{blue}

\usepackage{amssymb}

\usepackage[scale=0.8]{geometry}


% https://tex.stackexchange.com/questions/158803/add-a-table-of-contents-into-a-moderncv-document
\makeatletter
\newcommand\@pnumwidth{1.55em}
\newcommand\@tocrmarg{2.55em}
\newcommand\@dotsep{4.5}
\makeatother
\usepackage{titletoc}

\makeatletter
\setlength\columnsep{20pt}
\setcounter{tocdepth}{1}
\newcommand\contentsname{Contents}
\newcommand\tableofcontents{%
\addtocontents{toc}{\protect\setcounter{tocdepth}{0}}
\section*{\contentsname}
\addtocontents{toc}{\protect\setcounter{tocdepth}{2}}
    \@starttoc{toc}%
    }
\usepackage{titletoc}
\titlecontents*{section}[0pt]
  {}{}{}
  {, \itshape\thecontentspage}[\ \textbullet\ ][.]
\renewcommand\l@subsection[2]{}
\makeatother


\usepackage{xstring}
\usepackage{soul}
\setul{2pt}{1pt}
\def\FormatName#1{%
  \IfSubStr{#1}{Poon}{\ul{\textbf{#1}}}{#1}%
}


\makeatletter\renewcommand*{\bibliographyitemlabel}{\arabic{enumiv}.}\makeatother

\usepackage[resetlabels,labeled]{multibib}
\newcites{inprep,book,confpub}{{},{Books}, {Conference publications}}





\name{Art F.~Y.}{Poon}
%\title{Curriculum vitae}
\address{Health Sciences Addition, H422}{London, ON}{Canada N6A 5C1}
\phone[office]{(519) 661-2111 ext.~87978}
\email{apoon42@uwo.ca}
\social[github]{ArtPoon}
\social[twitter]{art\_poon}

\begin{document}

\makecvtitle
\tableofcontents


\section{Education}

\cventry{2000--2005}{PhD}{University of California, San Diego}{La Jolla}{Biology}{Thesis: Evolutionary consequences of the genetic environment on mutations in bacterial viruses (Dr.~Lin Chao)}
\cventry{1998--2000}{MSc}{University of British Columbia}{Vancouver}{Zoology}{Thesis:  The evolution of the genetic load caused by recurrent mutations in small populations: genetic context and demographic history (Dr.~Sarah P.~Otto)}
\cventry{1994--1998}{HonBSc}{University of Toronto}{Toronto}{Biology}{Thesis: Estimates of mutational variance in \textit{Drosophila melanogaster} (Dr.~David Houle)}


\section{Appointments}

\cventry{August 2016--present}{Assistant Professor}{Department of Pathology and Laboratory Medicine}{Western University}{}{Primary appointment.}

\cventry{June 2017--present}{Assistant Professor}{Department of Microbiology and Immunology}{Western University}{}{Cross-appointment.}

\cventry{November 2016--present}{Assistant Professor}{Department of Applied Mathematics}{Western University}{}{Cross-appointment.}

\cventry{March 2012--July 2016}{Assistant Professor, non-tenure track}{Department of Medicine}{University of British Columbia}{}{As an employee of Providence Health Care, I was held this honorary unpaid appointment to provide teaching, scholarly and service activities at the university.}

\cventry{October 2015--July 2016}{Senior Research Scientist}{}{BC Centre for Excellence in HIV/AIDS}{}{}

\cventry{October 2009 -- October 2015}{Associate Research Scientist, Bioinformatics}{}{BC Centre for Excellence in HIV/AIDS}{}{}

\cventry{July 2010 -- June 2015}{Adjunct Assistant Professor}{Faculty of Health Sciences}{Simon Fraser University}{}{}

\cventry{August 2008 -- August 2010}{Postdoctoral Fellow}{Department of Experimental Medicine}{University of British Columbia}{}{Supervisor: Dr.~P.~Richard Harrigan.}

\cventry{August 2005 -- July 2008}{Postdoctoral Fellow}{Antiviral Research Center}{University of California, San Diego}{}{Supervisor: Dr.~Simon Frost.}



\section{Awards and Recognitions}

\cventry{2013--2018}{CIHR New Investigator Award}{CIHR Priority Announcement: CHVI Vaccine Discovery and Social Research}{CAD\$300,000}{}{Project title: Applied phylogenetics for HIV prevention.}

\cventry{2012-2020}{Career Investigator Scholar Award}{Michael Smith Foundation for Health Research, St.~Paul's Hospital Foundation, and the Providence Health Care Research Institute}{CAD\$ 317,500}{}{Project title: Phylogenetic surveillance of the HIV epidemic in British Columbia.  In 2016, I declined the remainder of this award when I left British Columbia to accept a tenure-track appointment at Western University in Ontario.}

\cventry{2008-2010}{CIHR Fellowship}{CIHR Fellowships Award in the Area of Biomedical/Clinical HIV/AIDS Research}{CAD\$97,500}{}{}

\cventry{1998-2000}{Postgraduate Scholarship Award}{Natural Sciences and Engineering Research Council of Canada (NSERC)}{CAD\$37,000}{}{}




\section{Submitted manuscripts}

\vspace{-2em}
\nociteinprep{*}
\bibliographystyleinprep{custom}
\bibliographyinprep{inprep}



\section{Publications}


\renewcommand{\refname}{Peer-reviewed articles}
%\renewcommand{\bibliographyitemlabel}{{\arabic{enumiv}.}}

70 peer-reviewed articles; 33 as first or senior author.

\nocite{*}

\bibliographystyle{custom}
\bibliography{publications}



\vspace{1em}
\nocitebook{*}
\bibliographystylebook{custom}
\bibliographybook{book}

\vspace{1em}
\nociteconfpub{*}
\bibliographystyleconfpub{custom}
\bibliographyconfpub{confpub}



\section{Grants}

Only grants on which I am Principal Investigator/Applicant, Co-applicant or Co-investigator are listed.
These lists omit 5 grants in which I was included as a Collaborator or a Consultant.\\


\subsection {Pending}

\cventry{2018/4 -- 2021/3}
{Principal Applicant}
{CIHR Project Grant}
{Phylodynamics of HIV within hosts}
{In review}
{
This application is a resubmission of my Project Grant that received one year of bridge funding. 
Funding requested: \$500,000
}

\cventry{2018/4 -- 2021/3}
{Principal Applicant}
{CIHR Project Grant}
{Development, evaluation and implementation of genetic clustering methods for the real-time molecular surveillance of HIV outbreaks}
{In review}
{
This research proposal will support the development of a completely new method for genetic clustering in application to HIV molecular epidemiology; 
to evaluate clustering methods in the context of HIV prevention with computer simulations and the retrospective analysis of population databases; 
and to support the implementation and evaluation of real-time monitoring systems based on these methods in Uganda and Malaysia.\\
Funding requested: \$500,000
}

\cventry{2018/4 -- 2021/3}
{Co-Applicant}
{CIHR Project Grant}
{Integrated phylogenetic, molecular and functional analyses of the within-host latent HIV reservoir}
{In review}
{
Principal Applicant: Dr.~Zabrina Brumme\\
This application represents my ongoing collaboration with Drs.~Zabrina Brumme at Simon Fraser University, Canada. 
The objective is to refine and validate a molecular clock-based method to `date' HIV lineages from the latent reservoir, and to characterize associations between the age of the reservoir and immunological factors.
I played a key role in developing this method during my previous position in Vancouver.
My role will be to provide bioinformatics and phylogenetics expertise.\\
Funding requested: \$900,000
}

\cventry{2018/4 -- 2021/3}
{Co-Applicant}
{CIHR Project Grant}
{An Innovative and Highly Specific Approach for Detection of Salivary Biomarkers of Zika Fever}
{In review}
{
Principal Applicant: Dr.~Walter Sequeira\\
This application proposes to explore a new, uncharted area of diagnosing ZIKV by using
saliva and proteomics, based on mass spectrometry.
My role will be to provide bioinformatics support through the analysis of flavivirus genome sequence variation, to develop an optimized set of peptide probes.\\
Funding requested: \$450,000
}


\vspace{1em}
\subsection {Active}

\cventry{2017/9 -- 2019/3}
{Co-investigator}
{CIHR Operating Grant: Innovative Biomedical and Clinical HIV/AIDS Research}
{Developing the ACT-VEC as an HIV therapeutic vaccine and cure approach}
{Total funding: \$250,000}
{
Principal Applicant: Dr.~Eric Arts\\
The objective of this application is to develop vectors containing near-complete HIV genomes as a heterologous subtype-specific vaccine to induce the reactivation of infected CD4+ T cells from the latent reservoir, as a primary component of an HIV cure strategy.
My role will be to design and implement a bioinformatic/phylogenetic algorithm for selecting patient-derived HIV genomes to incorporate into the vaccines.
}

\cventry{2014/1 -- 2018/12}
{Co-investigator}
{CIHR Team Grant: HIV Cure Research}
{The Canadian HIV Cure Research Enterprise (CanCURE)}
{Total funding: \$8,760,000}
{
Principal Applicant: Dr.~Eric Cohen\\
The objective of this grant is to investigate the role of myeloid cells in maintaining the latent HIV reservoir.
My role is to provide bioinformatic support for the basic science component led by Drs.~Zabrina Brumme and Mark Brockman.
}

\cventry{2017/10 -- 2018/9}
{Principal Investigator}
{CIHR Project Grant: Bridge Funding}
{Phylodynamics of HIV within hosts}
{Total funding: \$100,000}
{
The main objectives of this grant are to develop a new branch of phylodynamics for studying the evolution of HIV within hosts, and to use these methods to test hypotheses on the maintenance of the latent HIV reservoir, and to study the compartmentalization of HIV evolution between the blood and genital tract.\\
}


\cventry{2016/10 -- 2018/9}
{Principal Investigator}
{Genome Canada/CIHR: Bioinformatics and Computational Biology}
{Kamphir: a versatile framework to fit models to tree shapes}
{Total funding: \$205,365}
{
The objective of this grant is the development and validation of a software module in R for fitting epidemic and diversification models to the shapes of phylogenetic trees, as an innovative means of studying the spread and evolution of viruses.
}

\cventry{2013/6 -- 2018/2}
{Co-investigator}
{National Institutes of Health (U.S.) R01}
{Seek and treat for the optimal prevention of HIV \& AIDS in BC}
{Total funding: \$530,331}
{
The objective of this grant is to support the implementation of strategic `test and treat' initiatives for HIV prevention in British Columbia.
My role was to provide bioinformatic support, including the phylogenetic and geographic analysis of HIV incidence trends based on the population HIV treatment database.\\
}


\cventry{2013/01 -- 2017/12}
{Co-applicant}
{Genome Canada: Genomics and Personalized Health}
{Viral and Human Genetic Predictors of Response to HIV Therapies}
{Total funding: \$4,100,000}
{
Principal Applicant: Dr.~P.~Richard Harrigan\\
The main objectives of this grant was to develop and validate a clinical next-generation sequencing (NGS) pipeline for diagnosing HIV drug resistance;
and to build a real-time monitoring system of transmitted HIV drug resistance.
My role was to design and implement both the clinical NGS pipeline and the real-time system, and eventually to supervise a team of developers to enhance and maintain these systems as lead developer.
}

\cventry{2012/9 -- 2017/8}
{Co-investigator}
{National Institutes of Health (U.S.) R01}
{Impacts of universal access to HIV/AIDS care among HIV+ injection drug users}
{Total funding: \$2,424,355}
{
Principal Investigator: Dr.~Evan Wood\\
This application requested continued funding support for the Vancouver Injection Drug User Study (VIDUS) and ACCESS cohorts in British Columbia, for the further evaluation of increasing access to antiretroviral treatment and harm reduction resources in Vancouver.
My role was to provide phylogenetic support for characterizing the effects of these initiatives on transmission rates.
}




\vspace{1em}

\subsection {Completed}

\cventry{2014/7 -- 2017/7}
{Co-investigator}
{Bill \& Melinda Gates Foundation Grant}
{Biostatistical, Computational Biology, and Mathematical Modeling for the Assessment of
Immune Correlates of Protection in the HVTN 701 and 702 Efficacy Trials of South Africa}
{Total funding: \$1,051,333}
{
Principal Investigator: Dr.~Peter B.~Gilbert
The purpose of this project was to develop, validate and apply phylogenetic methods to characterize the dynamics and adaptation of HIV in individuals who had become infected while participating in the vaccine efficacy trials HVTN 701 and 702.
My role was to implement some of the standard phylogenetic methods into a Python module to facilitate batch processing and testing.
}


\cventry{2012/10 -- 2015/9}
{Co-investigator}
{CIHR Operating Grant}
{Measuring mitochondrial aging, application to HIV infection and therapy}
{Total funding: \$330,150}
{
Principal Investigator: Dr.~H\'el\`ene C\^ot\'e\\
My role in this project was to assist in the analysis of next-generation sequence data that was collected to discriminate between the \textit{de novo} emergence and clonal expansion of mtDNA variants within patients undergoing antiretroviral treatment.
}

\cventry{2011/10 -- 2014/9}
{Co-investigator}
{CIHR Operating Grant}
{HIV adaptation to immune selection pressures: historic trends and future implications}
{Total funding: \$318,132}
{
Principal Investigator: Dr.~Zabrina Brumme\\
My role was to perform phylogenetic, molecular clock and ancestral reconstruction analyses on historic and modern HIV sequences.
}

\cventry{2011/10 -- 2014/9}
{Co-investigator}
{CIHR Operating Grant}
{Hepatitis C virus transmission dynamics among injection drug users}
{Total funding: \$336,569}
{
Principal Investigator: Jason Grebely\\
My role was to provide phylogenetic and genetic analysis software tools and expertise on their usage and interpretation of results in the context of HCV epidemics.
}

\cventry{2011/4 -- 2014/3}
{Principal Investigator}
{CIHR Operating Grant}
{Reconstructing within-host evolution of HIV-1 from next-generation sequencing data}
{Total funding: \$169,441}
{
The objective of this grant was to develop phylogenetic methods for the analysis of next-generation sequence data to reconstruct the evolution of HIV within hosts.
Specific objectives were to develop methods to estimate the date of HIV infection from sequence variation, and to map the emergence of specific variants associated with pathogenesis to the timeline of infection.
}


\cventry{2012/3 -- 2013/2}
{Principal Applicant}
{NIH R01 Administrative Supplement}
{Evaluating the natural history of injection drug use}
{Total funding: \$81,687}
{
Principal Investigator of parent grant: Dr.~Thomas Kerr\\
The purpose of this supplement was to develop methods to estimate the dates of HIV infection from bulk sequence chromatograms, to develop next-generation sequence (NGS) processing pipelines, and methods to estimate dates of HIV infection from NGS data sets.
}


\section {Presentations}

\subsection {Conferences}

\subsection {Invited seminars}




\section {Student/Postdoctoral supervision}


\section {Teaching}


\section {Academic service}



\end{document}




