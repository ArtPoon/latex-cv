% !TEX root = artpoon-CV.tex

\section{Source code}

Sorted in order of first commit date (left margin), in descending order.
Number of commits that are my own (total).\\

\parskip 1ex

\cventry{2022-01}
{duotang}
{\url{https://github.com/CoVaRR-NET/duotang}}
{Developer}
{73 (272) commits}
{duotang is an RMarkdown notebook (a text document with embedded R source code for analyzing data and generating graphics) summarizing Pillar 6's ongoing investigation of the evolution and epidemiology of SARS-CoV-2 variants of concern in Canada.}

\cventry{2020-10}
{OpenRDP}
{\url{https://github.com/PoonLab/OpenRDP}}
{Developer, package maintainer}
{35 (121) commits, 21$\bigstar$}
{
OpenRDP is an open-source re-implementation of the popular software package for detecting recombination in genome sequences, RDP.
OpenRDP was first implemented by Kaitlyn Wade, an undergraduate in the Department of Computer Science at Western University, as her fourth year thesis project under my supervision.
}


\cventry{2020-04}
{CoVizu}
{\url{https://github.com/PoonLab/covizu}}
{Original concept and lead developer}
{444 (1,002) commits, 43$\bigstar$}
{CoVizu is an open source project to develop a public interface to visualize the global diversity of SARS-CoV-2 genomes in near real time.
It consists of a Python back-end for analyzing the GISAID database (the largest SARS-CoV-2 database in the world), and a JavaScript front-end comprising an animated web interface hosted at \url{https://filogeneti.ca/CoVizu}.}


\cventry{2019-07}
{ggfree}
{\url{https://github.com/ArtPoon/ggfree}}
{Original concept and lead developer}
{83 (83) commits, 67$\bigstar$}
{ggplot2 is a popular R graphics package that is becoming synonymous with data visualization in R. The community of developers working within the ggplot2 framework have implemented some rather nice extensions as well. However, it is almost always possible for a visualization produced in ggplot2 to also be generated using the base graphics package in R. Long-time users of R who are accustomed to building plots with the latter may find the syntax of ggplot2 counter-intuitive and awkward. The overall purpose of ggfree is to make it easier to generate plots in the style of ggplot2 and its extensions, without ever actually using any ggplot2 code.}

\cventry{2018-02}
{BioID}
{\url{https://github.com/ArtPoon/BioID}}
{Original concept and lead developer}
{116 (116) commits, 16$\bigstar$}
{This repository holds the static files for a Jekyll site where I create materials for an undergraduate course on the bioinformatics of infectious disease.  These materials include interactive JavaScript animations, and are organized into an online textbook.  I have released all materials into the public domain under the Creative Commons license (CC-BY-SA 4.0).}



\cventry{2017-11}
{treeswithintrees}
{\url{https://github.com/PoonLab/twt}}
{Original concept and lead developer}
{150 (438) commits, 2$\bigstar$}
{treeswithintrees (twt) is an R package for the coalescent (reverse time) simulation of pathogen trees within host transmission trees.}


\cventry{2017-07}
{sierra-local}
{\url{https://github.com/PoonLab/sierra-local}}
{Original concept and senior developer}
{104 (214) commits, 4$\bigstar$}
{sierra-local is a Python 3 implementation of the Stanford University HIV Drug Resistance Database (HIVdb) Sierra web service for generating drug resistance predictions from HIV-1 sequence data. This Python package enables laboratories to run this prediction algorithm without needing to transmit patient data over the network, and confers full control over data provenance and security.}


\cventry{2016-10}
{Kaphi}
{\url{https://github.com/ArtPoon/Kaphi}}
{Original concept and lead developer}
{145 (632) commits, 4$\bigstar$}
{Phylodynamic inference is the fitting of models to the shape of a phylogenetic tree in order to reconstruct the historical processes that produced the tree.
Kaphi is an R package for fitting models to the shapes of phylogenetic trees.
It uses approximate Bayesian computation as a likelihood-free approach to fit models.
The shapes of simulated phylogenies are compared to the data with a kernel method.}



\cventry{2016-02}
{clmp}
{\url{https://github.com/PoonLab/clmp}}
{Original concept and lead R developer}
{79 (141) commits, 1$\bigstar$}
{Genetic clustering with Markov-modulated Poisson processes. 
clmp is an R extension, mostly written in C, for extracting genetic clusters from a phylogeny using a Markov-modulated Poisson process to model variation in branching rates. Our paper that describes and evaluates MMPP as a model-based clustering method for HIV epidemiology was recently accepted in PLOS Computational Biology.
}


\cventry{2013-03}
{Kive}
{\url{https://github.com/cfe-lab/Kive}}
{Original concept and lead developer (2013-2016)}
{172 (4,447) commits, 7$\bigstar$}
{Archival and automation of bioinformatic pipelines and data.}


\cventry{2013-01}
{MiCall}
{\url{https://github.com/cfe-lab/MiCall}}
{Original concept and lead developer}
{210 (2,408) commits, 11$\bigstar$}
{Pipeline for processing FASTQ data from an Illumina MiSeq to genotype human RNA viruses like HIV and hepatitis C.}


\cventry{2008-05}
{HyPhy}
{\url{https://github.com/veg/hyphy}}
{Major contributor, machine learning toolbox}
{68 (3,645) commits, 162$\bigstar$}
{
HyPhy is an open-source software package for the analysis of genetic sequences using techniques in phylogenetics, molecular evolution, and machine learning. 
It features a complete graphical user interface (GUI) and a rich scripting language for limitless customization of analyses. 
Additionally, HyPhy features support for parallel computing environments (via message passing interface (MPI)) and it can be compiled as a shared library and called from other programming environments such as Python and R. 
HyPhy is the computational backbone powering datamonkey.org. Additional information is available at hyphy.org.
Note most of my contributions to the source code preceded the development team's use of git for version control.
}



